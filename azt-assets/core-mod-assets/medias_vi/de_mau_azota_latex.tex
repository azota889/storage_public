\begin{name}
	{\tenchude}
	{\tendethi}
	{\tentruong}
	{\thoigian}
\end{name}

\Opensolutionfile{ansbook}[ans/ansbook1]
\Opensolutionfile{ans}[ans/dudoan03-TN]
% Các câu hỏi dạng trắc nghiệm 4 phương án
\begin{ex}
[NB] Diện tích xung quanh của hình nón có bán kính đáy $r$ và đường sinh $l$ là
\choice
{$\dfrac{1}{2}\pi rl$}
{\True $\pi rl$}
{$2\pi rl$}
{$\dfrac{1}{3}\pi r^2l$}
\end{ex}
\begin{ex}
[NB] Đường thẳng $d\colon y=-2x+1$ và parabol $(P)\colon y=x^2$ có số điểm chung là
\choice
{$3$}
{\True $2$}
{$1$}
{$0$}
\end{ex}
\begin{ex}
[NB] Biểu thức $\sqrt{-x+2024}$ xác định khi và chỉ khi
\choice
{$x<2024$}
{$x\ge 2024$}
{$x\ne 2024$}
{\True $x\le 2024$}
\end{ex}
\begin{ex}
[VD] Cho đường tròn $(O;R)$ và $(O';r)$ thoả mãn $R>r$ đồng thời $R-r<OO<R+r$. Số tiếp tuyến chung của hai đường tròn đó là
\choice
{$0$}
{\True $2$}
{$3$}
{$4$}
\end{ex}
\begin{ex}
[NB] Bất phương trình nào sau đây là bất phương trình bậc nhất một ẩn?
\choice
{$x-y<0$}
{\True $-x-\dfrac{3}{5}>0$}
{$x-5\le 0$}
{$x^2-4\le 0$}
\end{ex}
\begin{ex}
[NB] Chọn ngẫu nhiên một số tự nhiên có $2$ chữ số và chia hết cho $5$. Khi đó kết quả thuận lợi của biến cố là bao nhiêu?
\choice
{$17$}
{\True $18$}
{$19$}
{$20$}
\end{ex}
\begin{ex}
[NB] Số lượng tin nhắn một người nhận được vào các ngày làm việc trong tháng 6/2024 được ghi lại như sau:\\
Giá trị lớn nhất của mẫu số liệu là bao nhiêu? Tìm tần số của nó?
\begin{table}[h!]
\centering
\begin{tabular}{|c|c|c|c|c|c|c|c|c|c|c|c|c|c|c|}
\hline
6 & 2 & 8 & 2 & 4 & 10 & 6 & 6 & 10 & 8 & 6 & 2 & 8 \\
\hline
10 & 6 & 6 & 10 & 2 & 6 & 4 & 10 & 6 & 4 & 4 & 6 & 8 \\
\hline
\end{tabular}
\end{table}
\choice
{Giá trị lớn nhất của mẫu số liệu là $6$ với tần số là $9$}
{Giá trị lớn nhất của mẫu số liệu là $9$ với tần số là $6$}
{\True Giá trị lớn nhất của mẫu số liệu là $10$ với tần số là $5$}
{Giá trị lớn nhất của mẫu số liệu là $10$ với tần số là $4$}
\end{ex}
\begin{ex}
[NB] Hệ phương trình $\heva{&x-4y=6\\ &-4y=6+2x}$
\choice
{\True có nghiệm duy nhất}
{có 2 nghiệm}
{vô nghiệm}
{vô số nghiệm}
\end{ex}
\begin{ex}
[NB] Cho đường tròn $(O)$ có góc nội tiếp $\widehat{BAC}$ bằng $110^\circ$ ($B$ và $C$ thuộc đường tròn). Số đo của $\widehat{BOC}$ bằng
\choice
{$110^\circ$}
{$220^\circ$}
{$70^\circ$}
{\True $140^\circ$}
\end{ex}
\begin{ex}
[NB] Biết phương trình $3x^2-6x-9=0$ có hai nghiệm $x_1<x_2$. Khi đó biểu thức $\dfrac{x_2}{x_1}$ có giá trị là
\choice
{$\dfrac{1}{3}$}
{$-\dfrac{1}{3}$}
{$3$}
{\True $-3$}
\end{ex}
\begin{ex}
[NB] Cho tam giác $ABC$ vuông tại $A$ đường cao $AH$ và $HB=2cm$; $BC=8cm$. Độ dài cạnh $AC$ bằng.
\choice
{$4\sqrt{2}$ cm}
{$4$ cm}
{\True $4\sqrt{3}$ cm}
{$6$ cm}
\end{ex}
\begin{ex}
[NB] Số có căn bậc hai số học bằng 9 là
\choice
{$3$}
{$-3$}
{$3$ và $-3$}
{\True $81$}
\end{ex}
\begin{ex}
Cho tam giác $ABC$ vuông tại $A$. Biết $\sin C=\dfrac{3}{4}$; $BC=8$ cm. Độ dài cạnh $AB$ là
\choice
{\True $6$ cm}
{$4$ cm}
{$\dfrac{32}{3}$ cm}
{$3$ cm}
\end{ex}
\begin{ex}
[NB] Nếu $a, b$ và $c$ là các số bất kì và $a>b$ thì bất đẳng thức nào sau đây luôn đúng?
\choice
{$ac>bc$}
{$a^2>b^2$}
{$c-a>c-b$}
{\True $a+c>b+c$}
\end{ex}
\begin{ex}
[TH] Cho đường tròn $(O)$ và dây $AB$, $M$ là điểm chính giữa của cung nhỏ $AB$. Lấy điểm $C$ thuộc đoạn thẳng $AB$, đường thẳng $MC$ cắt $(O)$ tại $D$ khác $M$, biết độ dài $MC=9$ cm, $CD=16$ cm (tham khảo hình vẽ). Độ dài dây $MA$ bằng
\choice
{$12$ cm}
{\True $12$ cm}
{$25$ cm}
{$7$ cm}
\end{ex}
\begin{ex}
Biết $\sqrt{8-2\sqrt{7}}+3=a+b\sqrt{7}$ với ($a,b$ là các số nguyên). Khi đó $a^2+b^2$ bằng
\choice
{$4$}
{$3$}
{$1$}
{\True $5$}
\end{ex}
\begin{ex}
[TH] Bạn Minh gieo một con xúc xắc cân đối và đồng chất một số lần và ghi lại tần số, tần số tương đối số lần xuất hiện của mỗi mặt trong bảng sau.
\begin{table}[h!]
\centering
\begin{tabular}{|c|c|c|c|c|c|c|c|}
\hline
\textbf{Mặt} & 1 chấm & 2 chấm & 3 chấm & 4 chấm & 5 chấm & 6 chấm & \textbf{Tổng} \\
\hline
Tần số & 6 & 1 & 10 & 12 & 3 & 8 & $N=40$ \\
\hline
Tần số tương đối (\%) & 15 & 2,5 & 25 & 30 & 7 & 20 & 100 \\
\hline
\end{tabular}
\end{table}
Trong bảng số liệu trên có một số liệu không chính xác. Hãy tìm số liệu đó và sửa cho đúng?
\choice
{Giá trị $2$ chấm sai tần số tương đối, sửa lại $2\%$}
{\True Giá trị $5$ chấm sai tần số tương đối, sửa lại $7.5\%$}
{Giá trị $6$ chấm sai tần số tương đối, sửa lại $20.5\%$}
{Giá trị $4$ chấm sai tần số tương đối, sửa lại $30.5\%$}
\end{ex}
\begin{ex}
Tổng các nghiệm của phương trình $\sqrt{x^2-4x+4}=1$
\choice
{$1$}
{$2$}
{$-2$}
{\True $4$}
\end{ex}
\begin{ex}
Cho tam giác $ABC$ vuông tại $A$ đường cao $AH$ và $\dfrac{HB}{HC}=\dfrac{1}{2}$ cm. Tỉ số $\dfrac{AC}{AB}$ bằng
\choice
{$2$}
{$\dfrac{1}{\sqrt{2}}$}
{\True $\sqrt{2}$}
{$\dfrac{1}{4}$}
\end{ex}
\begin{ex}
[TH] Điểm $M$ có hoành độ bằng $1$ thuộc đồ thị hàm số $y=2x^2$. Gọi $A$ là điểm đối xứng của $M$ qua $Oy$, khi đó hoành độ của điểm $A$ là
\choice
{$1$}
{$2$}
{\True $-1$}
{$-2$}
\end{ex}
\begin{ex}
[TH] Tam giác $ABC$ vuông tại $A$ nội tiếp đường tròn $(O; 7,5cm)$. Biết $\dfrac{AB}{AC}=\dfrac{3}{4}$. Chu vi $\triangle ABC$ là
\choice
{$15\mathrm{~cm}$}
{\True $36\mathrm{~cm}$}
{$14,5\mathrm{~cm}$}
{$7.5\mathrm{~cm}$}
\end{ex}
\begin{ex}
Cho phương trình $x^2+3x-5=0$ có hai nghiệm $x_1, x_2$. Khi đó $x_1^2+x_2^2$ bằng
\choice
{$-1$}
{$1$}
{\True $19$}
{$14$}
\end{ex}
\begin{ex}
[VD] \immini[thm]
{
Một người đứng cách tòa nhà cao tầng $62m$, người đó nhìn lên đỉnh tòa nhà với phương nhìn tạo với phương ngang một góc bằng $35^\circ$ và mắt người này cách mặt đất $1.6m$. Chiều cao của tòa nhà (làm tròn đến chữ số thập phân thứ nhất) là
\choice
{$43,4m$}
{$37,2m$}
{$52,4m$}
{\True $45,0m$}
}
{
% TODO: \usepackage{graphicx} required
\includegraphics[scale=.65]{C:/texstudio-man/Anh/{7417F277-D044-44E2-B7BC-A2A33CBDE33E}}
}
\loigiai{
Ta có $\tan 35^\circ=\dfrac{AD}{62}$\\
$\Rightarrow AD=62.\tan 35^\circ=43.4m$\\
Vậy chiều cao tòa nhà là $AB=43.4+1.6=45.0$}
\end{ex}
\begin{ex}
[VD] Cho tứ giác $ABCD$ nội tiếp đường tròn $\left(O,R\right)$ có đường kính $AD=4~\mathrm{~cm};$ $AB=BC=1~\mathrm{~cm}$, khi đó $CD=$?
\choice
{\True $\dfrac{7}{2}~\mathrm{~cm}$}
{$\dfrac{7}{4}~\mathrm{~cm}$}
{$4~cm$}
{$2~cm$}
\loigiai{
\begin{center}\begin{tikzpicture}[>=stealth,line join=round,line cap=round,font=\footnotesize,scale=1]
\def\r{2.5}
\path
(0,0) coordinate (O)
(180:\r) coordinate (A)
(0:\r) coordinate (D)
(130:\r) coordinate (C)
(155:\r) coordinate (B)
(intersection of A--C and O--B) coordinate (E)
;
\foreach \d/\g in{A/180,O/-90,D/0,B/140,C/120,E/-90} \fill[] (\d) circle (1.2pt) node at ($(\d)+(\g:3mm)$){$\d$};
\draw (O) circle (\r) (C)--(A)--(D)--(C) (A)--(B)--(C)--(O)--(B);
\end{tikzpicture}\end{center}
Kẻ $AC$ cắt $BO$ tại $E$\\
Ta có bán kính $R=AD: 2=2(cm)$\\
Mà $BA=BC, OA=OC$\\
$\Rightarrow BO$ là trung trực $AC$\\
$\Rightarrow BO\perp AC$\\
Vậy $\triangle BEC$ vuông tại $E$\\
$\Rightarrow EC^2=BC^2-BE^2=1-BE^2$ (1)\\
Lại có $\triangle OEC$ vuông tại $E$\\
$\Rightarrow EC^2=OC^2-OE^2=4-(2-BE)^2$ (2)\\
Từ (1)(2) $\Rightarrow 1-BE^2=4-(2-BE)^2$\\
$\Rightarrow 1-BE^2-4+(2-BE)^2=0$\\
$\Rightarrow-4BE+1=0$\\
$\Rightarrow BE=\dfrac{1}{4}$\\
$\Rightarrow OE=2-\dfrac{1}{4}=\dfrac{7}{4}$\\
Ta có $\triangle ACD$ có $O, E$ lần lượt là trung điểm của $AC, AD$\\
$\Rightarrow OE$ là đường trung bình $\triangle ACD$.\\
$\Rightarrow OE=\dfrac{1}{2}CD$\\
$\Rightarrow CD=2OE=\dfrac{7}{2}(cm)$}
\end{ex}
\begin{ex}
Một hộp chứa ${4}$ thẻ gồm các số $9; 6; 4; 8$ (mỗi thẻ chỉ đánh một số). Rút ngẫu nhiên ${4}$ lần mỗi lần $1$ thẻ và không lần nào hoàn trả lại hộp, ghép các số của $4$ lần rút thẻ lại thành một số có $4$ chữ số với chữ số hàng nghìn là kết quả lần rút thứ nhất, chữ số hàng trăm là kết quả của lần rút thứ hai, chữ số hàng chục là kết quả của lần rút thứ $3$, chữ số hàng đơn vị là kết quả của lần rút cuối cùng. Xác suất để số tạo thành là số có chữ số ${6}$ xuất hiện ở hàng trăm và chữ số ${9}$ xuất hiện ở hàng đơn vị là:
\choice
{ \True ${\dfrac{1}{12}}$ }
{ ${\dfrac{2}{23}}$ }
{ ${\dfrac{1}{14}}$ }
{ ${\dfrac{1}{13}}$ }
\loigiai{
Số phần tử của không gian mẫu là ${24}$.\\
Số kết quả thuận lợi của biến cố là ${2}$.\\
Xác suất là ${\dfrac{1}{12}}$
}\end{ex}
\begin{ex}
Một hình trụ có thể tích $V=8\text{cm}^3$. Hỏi bán kính đáy bằng bao nhiêu để diện tích toàn phần của hình trụ đó là nhỏ nhất?
\choice
{\True $R=\sqrt[3]{\dfrac{4}{\pi}}\mathrm{~cm}$}
{$R=\sqrt[3]{\dfrac{8}{\pi}}\mathrm{~cm}$}
{$R=\sqrt[3]8\pi\mathrm{~cm}$}
{$R=\sqrt[3]4\pi\mathrm{~cm}$}
\loigiai{
Gọi $r,h$ lần lượt là bán kính đáy, chiều cao của hình trụ.\\
Ta có $V=\pi r^2h=8\Rightarrow h=\dfrac{8}{\pi r^2}$\\
$S_{tp}=2\pi rh+2\pi r^2=2\pi r.\dfrac{8}{\pi r^2}+2\pi r^2=\dfrac{16}{r}+2\pi r^2=\dfrac{8}{r}+\dfrac{8}{r}+2\pi r^2$\\
Áp dụng bất đẳng thức Cô-si cho ba số dương $\dfrac{8}{r};\dfrac{8}{r};2\pi r^2$ ta được\\
$\dfrac 8r+\dfrac 8r+2\pi r^2\geq 3\sqrt[3]{\dfrac 8r\cdot\dfrac 8r\cdot 2\pi r^2}=12\sqrt[3]{2\pi}$\\
Dấu bằng xảy ra khi $\dfrac{8}{r}=2\pi r^2\Leftrightarrow r^3=\dfrac{4}{\pi}\Leftrightarrow r=\sqrt[3]{\dfrac{4}{\pi}}$\\
}
\end{ex}
\Closesolutionfile{ans}
\Opensolutionfile{ans}[ans/dudoan03-DS]
% Các câu hỏi dạng đúng sai
\begin{ex}
Cho biểu thức $A=\sqrt{20}-2\sqrt{(1-\sqrt{5})^2}$ và $B=\dfrac{x+\sqrt{x}}{\sqrt{x}}+\dfrac{x-4}{\sqrt{x}+2}$.
Xét tính đúng, sai của các khẳng định sau:
\choiceTF
{Biểu thức $B$ xác định khi $x \ge 0, x \ne 4$} % Lời giải gốc là S (Sai)
{\True Rút gọn biểu thức $A$ ta được $A=2$} % Lời giải gốc là Đ (Đúng)
{Tại $x=3+2\sqrt{2}$ ta có $B=2\sqrt{2}$} % Lời giải gốc là S (Sai)
{Để giá trị biểu thức $A$ lớn hơn giá trị biểu thức $B$ thì $x < \dfrac{9}{4}$} % Lời giải gốc là S (Sai)
\loigiai{
\begin{itemchoice}
\itemch Biểu thức $B$ xác định khi $x \ge 0$. \\ Vậy mệnh đề sai.
\itemch $A=\sqrt{20}-2\sqrt{(1-\sqrt{5})^2} = 2\sqrt{5}-2|1-\sqrt{5}| = 2\sqrt{5}-2(\sqrt{5}-1) = 2\sqrt{5}-2\sqrt{5}+2 = 2$. \\ Vậy mệnh đề đúng.
\itemch Với $x \ge 0, x \ne 4$ ta có $B = \dfrac{\sqrt{x}(\sqrt{x}+1)}{\sqrt{x}} + \dfrac{(\sqrt{x}-2)(\sqrt{x}+2)}{\sqrt{x}+2} = \sqrt{x}+1 + \sqrt{x}-2 = 2\sqrt{x}-1$. \\
Với $x=3+2\sqrt{2} = 2+2\sqrt{2}+1 = (\sqrt{2}+1)^2$ thì $\sqrt{x} = \sqrt{(\sqrt{2}+1)^2} = |\sqrt{2}+1| = \sqrt{2}+1$. \\
Do đó $B = 2(\sqrt{2}+1) - 1 = 2\sqrt{2}+2-1 = 2\sqrt{2}+1$. \\ Vậy mệnh đề sai.
\itemch Để $A > B \Leftrightarrow 2 > 2\sqrt{x}-1 \Leftrightarrow 3 > 2\sqrt{x} \Leftrightarrow \sqrt{x} < \dfrac{3}{2} \Leftrightarrow x < \dfrac{9}{4}$. \\
Kết hợp với ĐKXĐ $x \ge 0, x \ne 4$, ta được $0 \le x < \dfrac{9}{4}$. \\ Vậy mệnh đề sai.
\end{itemchoice}
}
\end{ex}
\begin{ex}
\immini[thm]
{
Cho tam giác $ABC$ nhọn nội tiếp đường tròn $(O)$. Các đường cao $BD, CE$ của tam giác $ABC$ cắt nhau tại $H$. Gọi $M$ là trung điểm của $BC$.
Xét tính đúng, sai của các khẳng định sau:
\choiceTF
{$O$ là giao điểm ba đường phân giác của tam giác $ABC$} % Lời giải gốc là S (Sai)
{\True Bốn điểm $A, D, H, E$ cùng thuộc một đường tròn} % Lời giải gốc là Đ (Đúng)
{\True $\widehat{EDB} = \widehat{ECB}$} % Lời giải gốc là Đ (Đúng)
{\True $AH = 2OM$} % Lời giải gốc là Đ (Đúng)
}
{
\begin{tikzpicture}[>=stealth,line join=round,line cap=round,font=\footnotesize,scale=1]
\def\r{2.5}
\path
(0,0) coordinate (O)
(110:\r) coordinate (A)
(-40:\r) coordinate (C)
(220:\r) coordinate (B)
($(A)!(C)!(B)$) coordinate (E)
($(A)!(B)!(C)$) coordinate (D)
($(B)!(A)!(C)$) coordinate (A1)
(intersection of A--A1 and C--E) coordinate (H)
($(B)!(O)!(C)$) coordinate (M)
;
\draw (A)--(B)--(C)--(A)--(H) (B)--(D) (E)--(C) (O)--(M)
(O) circle (\r);
\foreach \d/\g in{A/90,B/-120,C/-50,O/90,E/150,D/60,H/-90,M/-90} \fill[] (\d) circle (1.2pt) node at ($(\d)+(\g:3mm)$){$\d$};
\end{tikzpicture}
}
\loigiai{
\begin{itemchoice}
\itemch $O$ là tâm đường tròn ngoại tiếp tam giác $ABC$, là giao điểm của ba đường trung trực của tam giác $ABC$. \\ Vậy mệnh đề sai.
\itemch Tam giác vuông $AEH$ và tam giác vuông $ADH$ cùng nội tiếp đường tròn đường kính $AH$. \\ Vậy bốn điểm $A, D, H, E$ cùng thuộc một đường tròn. \\ Vậy mệnh đề đúng.
\itemch Tam giác vuông $BEC$ và tam giác vuông $BDC$ cùng nội tiếp đường tròn đường kính $BC$. \\ Vậy tứ giác $BEDC$ nội tiếp đường tròn đường kính $BC$. \\ Do đó $\widehat{EDB} = \widehat{ECB}$ (hai góc nội tiếp cùng chắn cung $EB$). \\ Vậy mệnh đề đúng.
\itemch Kẻ đường kính $AF$ của đường tròn $(O)$. \\
Ta có $\widehat{ABF} = 90^\circ$ (góc nội tiếp chắn nửa đường tròn) hay $FB \perp AB$. \\
Vì $CE \perp AB$ (gt) $\Rightarrow FB \parallel CE$ \hfill(1). \\
Chứng minh tương tự ta được $FC \parallel BD$ (vì cùng vuông góc với $AC$).\hfill (2). \\
Từ (1) và (2) suy ra tứ giác $BFCH$ là hình bình hành. \\
Mà $M$ là trung điểm của $BC$ (gt) nên $M$ cũng là trung điểm của $HF$ (Tính chất đường chéo của hình bình hành). \\
Trong $\triangle AHF$ có: $O$ là trung điểm của $AF$ (vì $AF$ là đường kính), $M$ là trung điểm của $HF$ (cmt). \\
$\Rightarrow OM$ là đường trung bình của $\triangle AHF$. \\
Do đó $OM = \dfrac{1}{2}AH$ hay $AH = 2OM$. \\ Vậy mệnh đề đúng.
\end{itemchoice}
}
\end{ex}
\Closesolutionfile{ans}
\Opensolutionfile{ans}[ans/dudoan03-TLN]
% Các câu hỏi dạng trả lời ngắn
\begin{ex}
[TH] Cho $\triangle ABC$ nội tiếp đường tròn $(O;3 \text{ cm})$. Diện tích hình quạt tròn giới hạn bởi hai bán kính $OA, OC$ và cung nhỏ $AC$ khi $\widehat{ABC}=40^{\circ}$ là $a\pi \text{ cm}^2$. Tìm $a$?
\shortans{$2$}
\loigiai{
Vì $\triangle ABC$ nội tiếp đường tròn $(O;3 \text{ cm})$ nên đường tròn $(O;3 \text{ cm})$ ngoại tiếp $\triangle ABC$. \\
Xét $(O)$ có $\widehat{AOC}=2 \cdot \widehat{ABC}=2 \cdot 40^{\circ}=80^{\circ}$. \\
Diện tích hình quạt tròn giới hạn bởi hai bán kính $OA, OC$ và cung nhỏ $AC$ là: \\
$S=\dfrac{\pi \cdot 3^{2} \cdot 80}{360}=2\pi=a\pi$. \\
Vậy $a=2$.
}
\end{ex}
\begin{ex}
[VD] Phương trình $x^{2}-2mx+2m-3=0$ có hai nghiệm $x_{1}$, $x_{2}$ thỏa mãn $x_{1}^{2}+x_{2}^{2}=5$. Giá trị biểu thức $S=x_{1}+x_{2}-2x_{1}x_{2}$ bằng bao nhiêu?
\shortans{$5$}
\loigiai{
Phương trình $x^{2}-2mx+2m-3=0$ có $\Delta^{\prime}=m^{2}-2m+3=(m-1)^{2}+2>0$ với mọi $m$, nên phương trình luôn có hai nghiệm $x_{1}, x_{2}$. \\
Áp dụng định lí Viète ta có: $\begin{cases}x_{1}+x_{2}=2m\\ x_{1}x_{2}=2m-3\end{cases}$. \\
Theo đề bài: $x_{1}^{2}+x_{2}^{2}=5$ \\
$\Leftrightarrow (x_{1}+x_{2})^{2}-2x_{1}x_{2}=5$ \\
$\Leftrightarrow (2m)^{2}-2(2m-3)=5$ \\
$\Leftrightarrow 4m^{2}-4m+6=5$ \\
$\Leftrightarrow 4m^{2}-4m+1=0$ \\
$\Leftrightarrow (2m-1)^2=0$ \\
$\Leftrightarrow m=\dfrac{1}{2}$. \\
Khi đó: $\begin{cases}x_{1}+x_{2}=2 \cdot \dfrac{1}{2}=1\\ x_{1}x_{2}=2 \cdot \dfrac{1}{2}-3 = 1-3 = -2\end{cases}$. \\
Ta có: $S=x_{1}+x_{2}-2x_{1}x_{2}=1-2(-2)=1+4=5$.
}
\end{ex}
\begin{ex}
[VD] Hiện tại bạn Nam đã để dành được một số tiền là $800\,000$ đồng. Bạn Nam đang có ý định mua một chiếc xe đạp giá $2\,000\,000$ đồng, nên hàng ngày bạn Nam đều để dành cho mình $20\,000$ đồng. Sau ít nhất bao nhiêu ngày kể từ ngày bắt đầu tiết kiệm thì bạn Nam có thể mua được chiếc xe đạp đó?
\shortans{$60$}
\loigiai{
Số tiền Nam cần tiết kiệm thêm là: $2\,000\,000 - 800\,000 = 1\,200\,000$ đồng. \\
Số ngày Nam cần tiết kiệm là: $\dfrac{1\,200\,000}{20\,000} = 60$ (ngày).
}
\end{ex}
\begin{ex}
[VDC] Một viên than tổ ong có dạng hình trụ, đường kính đáy là $114 \text{ mm}$, chiều cao là $100 \text{ mm}$. Viên than này có $20$ lỗ "tổ ong" hình trụ có trục song song với trục của viên than, mỗi lỗ có đường kính $12 \text{ mm}$. Tính thể tích nhiên liệu được nén trong mỗi viên than tổ ong (làm tròn đến $\text{cm}^3$). Lấy $\pi \approx 3,14$.
\shortans{$794$}
\loigiai{
Đổi đơn vị: $114 \text{ mm} = 11,4 \text{ cm}$; $100 \text{ mm} = 10 \text{ cm}$; $12 \text{ mm} = 1,2 \text{ cm}$. \\
Bán kính đáy của viên than là: $R = \dfrac{11,4}{2} = 5,7 \text{ cm}$. \\
Thể tích toàn bộ viên than khi chưa có lỗ là: $V_{\text{trụ}} = \pi R^2 h \approx 3,14 \cdot (5,7)^2 \cdot 10 \approx 3,14 \cdot 32,49 \cdot 10 \approx 1020,186 \text{ cm}^3$. \\
Bán kính một lỗ "tổ ong" là: $r = \dfrac{1,2}{2} = 0,6 \text{ cm}$. \\
Thể tích một lỗ "tổ ong" là: $V_{\text{lỗ}} = \pi r^2 h \approx 3,14 \cdot (0,6)^2 \cdot 10 = 3,14 \cdot 0,36 \cdot 10 = 11,304 \text{ cm}^3$. \\
Tổng thể tích của $20$ lỗ "tổ ong" là: $V_{\text{20 lỗ}} = 20 \cdot V_{\text{lỗ}} \approx 20 \cdot 11,304 = 226,08 \text{ cm}^3$. \\
Thể tích nhiên liệu được nén trong mỗi viên than là: $V = V_{\text{trụ}} - V_{\text{20 lỗ}} \approx 1020,186 - 226,08 = 794,106 \text{ cm}^3$. \\
Làm tròn đến $\text{cm}^3$, thể tích nhiên liệu là $794 \text{ cm}^3$.
}
\end{ex}
\begin{ex}
[VDC] Giá trị nhỏ nhất của biểu thức $A=\sqrt{x-2-2\sqrt{x-3}}+\sqrt{x+1-4\sqrt{x-3}}$ (với $x\ge3$) là bao nhiêu?
\shortans{$1$}
\loigiai{
Với $x \ge 3$, ta có: \\
$A = \sqrt{(x-3)-2\sqrt{x-3}+1} + \sqrt{(x-3)-4\sqrt{x-3}+4}$ \\
$A = \sqrt{(\sqrt{x-3}-1)^2} + \sqrt{(\sqrt{x-3}-2)^2}$ \\
$A = |\sqrt{x-3}-1| + |\sqrt{x-3}-2|$. \\
Đặt $t = \sqrt{x-3}$, với $x \ge 3$ thì $t \ge 0$. \\
$A = |t-1| + |t-2| = |t-1| + |2-t|$. \\
Áp dụng bất đẳng thức $|a|+|b| \ge |a+b|$, ta có: \\
$A = |t-1| + |2-t| \ge |(t-1) + (2-t)| = |1| = 1$. \\
Dấu "=" xảy ra khi $(t-1)(2-t) \ge 0$. \\
$\Leftrightarrow (t-1 \ge 0 \text{ và } 2-t \ge 0) \text{ hoặc } (t-1 \le 0 \text{ và } 2-t \le 0)$ \\
$\Leftrightarrow (t \ge 1 \text{ và } t \le 2) \text{ hoặc } (t \le 1 \text{ và } t \ge 2)$ \\
Trường hợp $t \le 1$ và $t \ge 2$ là vô lý. \\
Vậy $1 \le t \le 2$. \\
$\Leftrightarrow 1 \le \sqrt{x-3} \le 2$ \\
$\Leftrightarrow 1 \le x-3 \le 4$ \\
$\Leftrightarrow 4 \le x \le 7$. \\
Giá trị này thỏa mãn điều kiện $x \ge 3$. \\
Vậy giá trị nhỏ nhất của $A$ là $1$, xảy ra khi $4 \le x \le 7$.
}
\end{ex}
\begin{ex}
[VDC] Một quả cầu gỗ có bán kính là $R=5 \text{ cm}$ được đặt trên một cái đế bằng gỗ có dạng một nửa mặt cầu bán kính bằng $\dfrac{R}{2}$. Hãy tính khoảng cách từ mặt đất đến điểm cao nhất của mặt cầu gỗ. (Kết quả làm tròn đến hàng phần mười).
% TODO: \usepackage{graphicx} required
\begin{center}
\begin{tabular}{cc}
\includegraphics[scale=0.4]{C:/texstudio-man/Anh/_147423D3-AC3D-4430-873A-6F3D7C12C9D7_-removebg-preview} \hspace*{2cm}&  \begin{tikzpicture}[>=stealth,line join=round,line cap=round,font=\footnotesize,scale=1]
\def\r{1.5}
\path
(0,0) coordinate (O)
(-60:\r) coordinate (B)
(240:\r) coordinate (A)
(90:\r) coordinate (E)
(intersection of A--B and O--E) coordinate (H)
($(H)+(-90:\r/2)$) coordinate (D)
;
\draw (O) circle (\r) (A)--(O)--(B)--(A) (D)--(E);
\draw[black] (A) arc (-180:0:\r/2);
\foreach \d/\g in{E/90,O/40,A/220,B/-40,D/-90,H/50} \fill[] (\d) circle (1.2pt) node at ($(\d)+(\g:3mm)$){$\d$};
\end{tikzpicture}
\end{tabular}
\end{center}
\shortans{$11,8$}
\loigiai{
Khoảng cách từ mặt đất đến điểm cao nhất của mặt cầu gỗ là đoạn $ED$\\
Gọi $H$ là trung điểm của $AB$ và dây $AB$ không đi qua tâm $O$ $OH\perp AB$ (liên hệ giữa đường kính và dây cung)\\
Ta có: $AB=2\cdot \dfrac{R}{2}=R$ (Vì $AB$ là đường kính)\\
Xét $\triangle OAB$ có: $OA=OB=AB=R\Rightarrow \triangle OAB$ đều\\
Suy ra $\widehat{OAB}=60^{\circ}$ hay $\widehat{OAH}=60^{\circ}$\\
Xét $\triangle AHO$ vuông tại $H$\\
Suy ra $\operatorname{Sin} OAH=\dfrac{OH}{OA}$ (tỉ số lượng giác góc nhọn)\\
Suy ra $OH=OA\cdot \operatorname{Sin} 60^{\circ}=R\cdot \dfrac{\sqrt{3}}{2}$\\
Ta có: $ED=OE+OH+HD=R+\dfrac{R\sqrt{3}}{2}+\dfrac{R}{2}=\dfrac{R(3+\sqrt{3}}{2}=\dfrac{5(3+\sqrt{3})}{2}=11,8$\\
Vậy chiều cao của quả cầu gỗ là $11,8\mathrm{~cm}$
}
\end{ex}
\Closesolutionfile{ans}
\Closesolutionfile{ansbook}

% Bảng đáp án
\indapan{dudoan03-TN}{dudoan03-DS}{dudoan03-TLN}
